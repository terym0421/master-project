\begin{thebibliography}{99}
  
  \bibitem{positivity} W. Graham, Positivity in equivariant Schubert calculus. Duke Math. J. 109 (2001).
  \bibitem{fulton young tableaux} W. Fulton, Young Tableaux: With Applications to Representation Theory and Geometry. Cambridge: Cambridge University Press, (1996).
  \bibitem{tu equivariant} L. W. Tu, Introductory Lectures on Equivariant Cohomology, Princeton University Press, (2020).
  \bibitem{knutson tao} A. Knutson and T.~C. Tao, Puzzles and (equivariant) cohomology of Grassmannians, Duke Math. J. {\bf 119}, no.~2, 221--260  (2003).
  \bibitem{thomas yong} H. Thomas and A.~T.~F. Yong, Equivariant Schubert calculus and jeu de taquin, Ann. Inst. Fourier (Grenoble) {\bf 68}, no.~1, 275--318 (2018). 
  \bibitem{schutzenberger}M.-P. Schützenberger, La correspondance de Robinson, in Combinatoire et
  repr\'esentation du groupe sym\'etrique (Actes Table Ronde CNRS, Strasbourg, 1976),
  Lecture Notes in Math., vol. 579, Springer, 59--113, (1977).
  \bibitem{milnor construction} J. Milnor, Construction of Universal Bundles, II. Annals of Mathematics 63, no. 3: 430--36, (1956).
  \bibitem{hatcher Top} A. Hatcher, Algebraic Topology, Cambridge University Press, (2001).
  \bibitem{hatcher VB} A. Hatcher, Vector Bundles and K-theory ver2.2, November 2017. 
  
  https://pi.math.cornell.edu/\textasciitilde hatcher/VBKT/VB.pdf, (accessed Dec 31, 2024).
  \bibitem{GKM} M. Goresky, R. Kottwitz and R. MacPherson, Equivariant cohomology, Koszul duality, and the localization theorem. Invent math 131, 25--83 (1997).
  \bibitem{steenrod} N. Steenrod, The Topology of Fibre Bundles. Princeton Math. Series 14,
  Princeton Univ. Press, Princeton, New Jersey, (1951).
  \bibitem{Kreiman} V. Kreiman, Equivariant Littlewood-Richardson skew tableaux. arXiv:0706.3738v2 [math.AG], (2007).
  \bibitem{representation} A. W. Knapp, Representation Theory of Semisimple Groups: An Overview Based on Examples (PMS-36). REV-Revised, Princeton University Press, (1986). 
  \bibitem{atiyah-bott} M.F. Atiyah and R. Bott.
  The moment map and equivariant cohomology.
  Topology,
  Volume 23, Issue 1, 1--28, (1984).
  \bibitem{berline-vergne} N. Berline and M. Vergne, Classes caract\'eristiques \'equivariantes. Formule
  de localisation en cohomologie \'equivariante, C. R. Acad. Sci. Paris, S\'erie
  I, t. 295: 539--540 (1982).
  \bibitem{cartan2}H. Cartan, La transgression dans un groupe de Lie et dans un espace fibr\'e principal. Colloque de
  topologie, C.B.R.M., Bruxelles, 57--71 (1950).
\end{thebibliography}