\section{同変コホモロジー}
\subsection{Borel構成}

$G$をコンパクトLie群とする.以下考える位相空間はすべて$CW$複体であるとする.本論文では特に断らない限りコホモロジーの係数はすべて$\integer$である.

\begin{defin}
  次の2つの条件を満たす主$G$束$f\colon EG\rightarrow BG$をuniversal $G$ bundleという.
  \begin{enumerate}
    \item 任意の主$G$束$h\colon E\rightarrow B$に対して連続写像$h\colon B\rightarrow EB$が存在して$E\simeq h^*EG$が成り立つ.
    \item $h_0,h_1\colon B\rightarrow EB$に対して$h_0^*EG\simeq h_1^*EG$ならば$h_0,h_1$はホモトピックである.
  \end{enumerate}
\end{defin}

\begin{theo}(Steenrod \cite{steenrod})
  主$G$束$E\rightarrow B$がuniversal $G$ bundleであるための必要十分条件は$E$が弱可縮であることである.
\end{theo}

\begin{eg}
  $S^\infty$\footnote{入る位相について明記?}は可縮であるので,$T=\complex^*$に対して$S^\infty\rightarrow \mathbb{CP}^\infty$はuniversal $T$ bundleである.
\end{eg}

$G$がコンパクトLie群の場合,universal $G$ bundleは直交群$O(k)$のuniversal bundleから構成できる.

$V_k(\real^n)$を$\real^n$の$k$個の正規直交なベクトル$(v_1,\cdots,v_k)$のなす空間とし,$\text{Gr}_k(\real^n)$を$\real^n$の$k$次元部分空間全体のなす空間とする.$V_k(\real^n)$をStiefel多様体,$\text{Gr}_k(\real^n)$をGrassmann多様体という.

\begin{prop}
  $V_k(\real^n)\rightarrow \text{Gr}_k(\real^n)$は主$O(k)$束であり,$\pi_i(V_k(\real^n))=0$ for $i=0,1,\cdots,n-k-1$である.
\end{prop}

\begin{proof}
  
\end{proof}

$V_k(\real^\infty)=\varinjlim V_k(\real^n)$, $\text{Gr}_k(\real^\infty)=\varinjlim \text{Gr}_k(\real^n)$とする.

\begin{prop}\label{universal O(k) bundle}
  $V_k(\real^\infty)\rightarrow \text{Gr}_k(\real^\infty)$はuniversal $O(k)$ bundleである.
\end{prop}

\begin{proof}
  
\end{proof}

\begin{theo}\label{universal bundle for subgroup}
  $G$をLie群,$H$を$G$の閉部分群とする.$EG\rightarrow EG/H$はuniversal $H$ bundleである.
\end{theo}

\begin{proof}
  
\end{proof}

任意のコンパクトLie群$G$は十分大きい$n$に対して$O(n)$に埋め込めることが知られている().従って定理\ref{universal O(k) bundle}と定理\ref{universal bundle for subgroup}から,$V_n(\real^\infty)\rightarrow V_n(\real^\infty)/G$はuniversal $G$ bundleである.

\begin{defin}
  $G$が$X$に左から作用しているとき,$G$の$X\times EG$への左作用を
  \[
  g(x, e):=(gx, e\inv{g}) \quad \text{for } g\in G, x\in X, e\in EG 
  \]
  によって定める.$X\times_GEG\colon=(X\times EG)/G$としこれを$X$のhomotopy quotient という.このとき
  $H^*_G(X):=H^*(X\times_GEG)$を$X$の$G$同変コホモロジーという.
\end{defin}

\begin{eg}
  1点集合$\text{pt}$の$G$同変コホモロジーは
  \[
  \text{pt}\times_GEG=(\text{pt}\times EG)/G\approx BG
  \]
  より
  \[
  H^*_G(\text{pt})\simeq H^*(BG)
  \]
  である。よって
  \[
  H^*_{\complex^*}(\text{pt})\simeq H^*(\mathbb{CP}^\infty)\simeq \integer[y]
  \]
\end{eg}


$X$の$G$同変コホモロジーは主$G$束$EG\rightarrow BG$の取り方に拠らないことを示そう.
写像$p\colon X\times EG\rightarrow X\times_GEG$と $p_X\colon X\times_GEG\rightarrow BG$を
\begin{align*}
  &p(x, e):=[x, e]\\
  &p_X([x, e]):=\pi(e)
\end{align*}
によって定める.

\begin{prop}
  \:
  \begin{enumerate}
    \item $p\colon X\times EG\rightarrow X\times_GEG$は主$G$束である
    \item $p_X\colon X\times_GEG\rightarrow BG$は$X$をファイバーとするファイバー束である
  \end{enumerate}
\end{prop}

\begin{proof}
  \begin{enumerate}
    \item $EG\rightarrow BG$は主$G$束であるので,
  \end{enumerate}
\end{proof}

連続写像$f\colon X\rightarrow Y$がホモトピー群の同型
\[
f_*\colon\pi_q(X, x)\rightarrow \pi_q(Y, f(x))\quad\text{for } x\in X, q>0 
\]
を誘導するとき、$f$を弱ホモトピー同値\footnote{弱可縮よりも先に書く?}という。

\begin{lemm}
  $E$を弱可縮な$G$空間とし、$P\rightarrow B$を主$G$束とする。このとき$(E\times P)/G\rightarrow B$は弱ホモトピー同値である。
\end{lemm}

\begin{proof}
  
\end{proof}

\begin{theo}\label{existance weak homotopy eq}
  $M$を$G$空間, $E\rightarrow B$, $E'\rightarrow B'$を主$G$束で$E,E'$はともに弱可縮であるとする。このとき弱ホモトピー同値$E\times_GM\rightarrow E'\times_GM$が存在する
\end{theo}

\begin{proof}
  
\end{proof}

\begin{theo}(Hatcher \cite{hatcher})\label{weak homotopy to cohomology}
  弱ホモトピー同値$f\colon X\rightarrow Y$は同型$f^*\colon H^*(Y)\rightarrow H^*(X)$を誘導する
\end{theo}

定理\ref{existance weak homotopy eq}と定理\ref{weak homotopy to cohomology}より, $EG\rightarrow BG$, $EG'\rightarrow BG'$が事実\ref{universal G bundle}の主$G$束であるとき、
\[
H^*(X\times_GEG)\simeq H^*(X\times_GEG')
\]
であることがわかる。

\subsection{$H^*_G(X)$の代数的構造}

$X, Y$を$G$空間, $f\colon X\rightarrow Y$を$G$写像とする。$f_G\colon X\times_GEG\rightarrow Y\times_GEG$を
\[
f_G([x, e])=[f(x), e]
\]
によって定めると$f_G$はwell-definedな連続写像となる。したがって$f_G$は同変コホモロジーの準同型
\[
f_G^*\colon H^*_G(Y)\rightarrow H^*_G(X)
\]
を誘導する。通常のコホモロジーの関手性と同様、同変コホモロジーも関手性をもつ
\begin{prop}\:
  \begin{enumerate}
    \item $(\id{X})_G^*=\id{H^*_G(X)}$
    \item $f\colon X\rightarrow Y$, $g\colon Y\rightarrow Z$に対して$(g\circ f)_G^*=(f_G^*)\circ(g_G^*)$
  \end{enumerate}
\end{prop}

任意の$G$空間$X$に対して、1点集合$\text{pt}$への自明な$G$写像は,準同型$H^*(BG)\simeq H^*_G(\text{pt})\rightarrow H^*_G(X)$を誘導するから、$H^*_G(X)$は$H^*(BG)$代数の構造を持つ.


\subsection{同変Euler類}



\begin{prop}
  $G$をコンパクト連結Lie群,$M$を向き付けられたなめらかなコンパクト多様体,$G$は$M$に作用し$N\subset M$を$G$-invariantな部分多様体で$\codim N=k$とする.$N$の定めるequivariant cohomology classを$[N]\in H^k_G(M)$とおく.このとき$[N]|_p\in H^k_G(M)$は$N$のnormal bundleの$p$におけるウェイトの積に等しい
\end{prop}

\begin{proof}
  
\end{proof}




\subsection{Weil/Cartanモデル}





\subsection{localization theorem}

