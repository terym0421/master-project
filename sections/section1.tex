\section{同変コホモロジー}
\subsection{Borel構成}

$X$を位相空間, $G$をコンパクトLie群とする.

\begin{fact}\label{universal G bundle}
  ある主$G$束$\pi\colon EG\rightarrow BG$が存在して,任意の主$G$束$E\rightarrow X$に対してある連続写像$f\colon X\rightarrow BG$があって$E=f^*(EG)$がなりたつ.さらに$EG$は弱可縮であり,$G$は$EG$に自由に(右から)作用する.
\end{fact}

\begin{eg}
  $T=\complex^*$に対して$S^\infty\rightarrow \mathbb{CP}^\infty$は事実\ref{universal G bundle}の主$G$束である。
\end{eg}

\begin{defin}
  $G$が$X$に左から作用しているとき,$G$の$X\times EG$への左作用を
  \[
  g(x, e) := (gx, e\inv{g}) \quad \text{for } g\in G, x\in X, e\in EG 
  \]
  によって定める.$X\times_GEG\colon=(X\times EG)/G$とし,これを$X$のhomotopy quotient という.このとき
  $H^*_G(X)\colon=H^*(X\times_GEG)$を$X$の$G$同変コホモロジーという.
\end{defin}

\begin{eg}
  1点集合$\text{pt}$の$G$同変コホモロジーは
  \[
  \text{pt}\times_GEG=(\text{pt}\times EG)/G\approx BG
  \]
  より
  \[
  H^*_G(\text{pt})\simeq H^*(BG)
  \]
  である。よって
  \[
  H^*_{C^*}(\text{pt})\simeq H^*(\mathbb{CP}^\infty)\simeq \integer[y]
  \]
\end{eg}


$X$の$G$同変コホモロジーは主$G$束$EG\rightarrow BG$の取り方に拠らないことを示そう.
写像$p\colon X\times EG\rightarrow X\times_GEG$と $p_X\colon X\times_GEG\rightarrow BG$を
\begin{align*}
  &p(x, e):=[x,e]\\
  &p_X([x,e]):=\pi(e)
\end{align*}
によって定める.

\begin{prop}
  \begin{enumerate}
    \item $p\colon X\times EG\rightarrow X\times_GEG$は主$G$束である
    \item $p_X\colon X\times_GEG\rightarrow BG$は$X$をファイバーとするファイバー束である
  \end{enumerate}
\end{prop}

\begin{proof}
  \begin{enumerate}
    \item $EG\rightarrow BG$は主$G$束であるので,
  \end{enumerate}
\end{proof}

連続写像$f\colon X\rightarrow Y$がホモトピー群の同型
\[
f_*\colon\pi_q(X, x)\rightarrow \pi_q(Y, f(x))\quad\text{for} x\in X, q>0 
\]
を誘導するとき、$f$を弱ホモトピー同値という。

\begin{lemm}
  $E$を弱可縮な$G$空間とし、$P\rightarrow B$を主$G$束とする。このとき$(E\times P)/G\rightarrow B$は弱ホモトピー同値である。
\end{lemm}

\begin{theo}\label{existance weak homotopy eq}
  $M$を$G$空間, $E\rightarrow B$, $E'\rightarrow B'$を主$G$束で$E,E'$はともに弱可縮であるとする。このとき弱ホモトピー同値$E\times_GM\rightarrow E'\times_GM$が存在する
\end{theo}

\cite{hatcher}より、次が成り立つ 
\begin{fact}\label{weak homotopy to cohomology}
  弱ホモトピー同値$f\colon X\rightarrow Y$は同型$f^*\colon H^*(Y)\rightarrow H^*(X)$を誘導する
\end{fact}

定理\ref{existance weak homotopy eq}と事実\ref{weak homotopy to cohomology}より, $EG\rightarrow BG$, $EG'\rightarrow BG'$が事実\ref{universal G bundle}の主$G$束であるとき、
\[
H^*(X\times_GEG)\simeq H^*(X\times_GEG')
\]
であることがわかる。

\subsection{$H^*_G(X)$の代数的構造}

$X, Y$を$G$空間, $f\colon X\rightarrow Y$を$G$写像とする。$f_G\colon X\times_GEG\rightarrow Y\times_GEG$を
\[
f_G([x, e])=[f(x), e]
\]
によって定めると$f_G$はwell-definedな連続写像となる。したがって$f_G$は同変コホモロジーの準同型
\[
f_G^*\colon H^*_G(Y)\rightarrow H^*_G(X)
\]
を誘導する。通常のコホモロジーの関手性と同様、同変コホモロジーも関手性をもつ
\begin{prop}\:
  \begin{enumerate}
    \item $(\id{X})_G^*=\id{H^*_G(X)}$
    \item $f\colon X\rightarrow Y$, $g\colon Y\rightarrow Z$に対して$(g\circ f)_G^*=(f_G^*)\circ(g_G^*)$
  \end{enumerate}
\end{prop}

任意の$G$空間$X$に対して、1点集合$\text{pt}$への自明な$G$写像は,準同型$H^*(BG)\simeq H^*_G(\text{pt})\rightarrow H^*_G(X)$を誘導するので、$H^*_G(X)$は$H^*(BG)$代数の構造を持つことがわかる

\subsection{Weil/Cartanモデル}





\subsection{localization theorem}
