\section{同変コホモロジー}
\subsection{universal bundle}

$G$をコンパクトLie群とする.以下考える位相空間はすべてCW複体であるとする.本論文では特に断らない限りコホモロジーの係数はすべて$\integer$である.

\begin{defin}
  次の2つの条件を満たす主$G$束$f\colon EG\rightarrow BG$をuniversal $G$ bundleという.
  \begin{enumerate}
    \item 任意の主$G$束$E\rightarrow B$に対して連続写像$h\colon B\rightarrow EB$が存在して$E\simeq h^*EG$が成り立つ.
    \item $h_0,h_1\colon B\rightarrow EB$に対して$h_0^*EG\simeq h_1^*EG$ならば$h_0,h_1$はホモトピックである.
  \end{enumerate}
\end{defin}

\begin{defin}
  連続写像$f\colon X\rightarrow Y$が弱ホモトピー同値であるとは,$f$が同型$f_*\colon\pi_q(X)\rightarrow \pi_q(Y)$をすべての$q=0,1,\cdots$に対して誘導することをいう.特に,$1$点への自明な写像$X\rightarrow \text{pt}$が弱ホモトピー同値になるような空間$X$を弱可縮であるという.
\end{defin}

\begin{theo}(Steenrod \cite{steenrod})\label{steenrod}
  主$G$束$E\rightarrow B$がuniversal $G$ bundleであるための必要十分条件は$E$が弱可縮であることである.
\end{theo}

\begin{eg}
  $S^\infty$\footnote{入る位相について明記?}は可縮であるので,$T=S^1$に対して$S^\infty\rightarrow \mathbb{CP}^\infty$はuniversal $T$ bundleである.
\end{eg}

$G$がコンパクトLie群の場合,universal $G$ bundleは直交群$O(k)$のuniversal bundleから構成できる.

$V_k(\real^n)$を$\real^n$の$k$個の正規直交なベクトル$(v_1,\cdots,v_k)$のなす空間とし,$\text{Gr}_k(\real^n)$を$\real^n$の$k$次元部分空間全体のなす空間とする.$V_k(\real^n)$をStiefel多様体,$\text{Gr}_k(\real^n)$をGrassmann多様体という.$(v_1,\cdots,v_k)\in V_k(\real^n)$を$n\times k$行列とみなすことで,$O(k)$は$V_k(\real^n)$に自由に右作用する.また$(v_1,\cdots,v_k)\in V_k(\real^n)$に対して,$v_1,\cdots,v_k$の生成する部分空間を対応させることで写像$V_k(\real^n)\rightarrow \text{Gr}_k(\real^n)$が定まる.

\begin{prop}
  $V_k(\real^n)\rightarrow \text{Gr}_k(\real^n)$は主$O(k)$束である.
\end{prop}

\begin{proof}
  簡単のため,$k=2,n=4$の場合に示す.$U_{1,2}\subset \text{Gr}_2(\real^4)$を
  \begin{equation}\label{basis}
    \left(\begin{array}{c}
      1\\0\\ * \\ *
    \end{array}\right),\quad \left(\begin{array}{c}
      0\\1\\ * \\ *
    \end{array}\right)
  \end{equation}
  の形のベクトルで生成される部分空間のなす集合とする.ここで$*$は任意の実数である.(\ref{basis})にSchmidtの直交化法を施して得られるベクトルを$v_1,v_2$とし,(\ref{basis})に$(v_1,v_2)$を対応させれば,切断$s_{1,2}\colon U_{1,2}\rightarrow V_k(\real^n)$
  得られる.$\phi_{1,2}\colon U_{1,2}\times O(2)\rightarrow V_{2}(\real^4)|_{U_{1,2}}$を
  \[
  \phi_{1,2}(\left\langle
    \left(\begin{array}{c}
      1\\0\\ * \\ *
    \end{array}\right),\:\left(\begin{array}{c}
      0\\1\\ * \\ *
    \end{array}\right)
  \right\rangle, P) = (v_1,v_2)P
  \]
  とすれば$\phi_{1,2}$は像への同相であり,$O(2)$の右作用と可換である.
  
  同様に$U_{i,j},s_{i,j},\phi_{i,j}$を$1\leq i<j\leq 4$に対して定義すれば$G$の作用と可換な局所自明化が得られる.
\end{proof}

\begin{prop}
  $V_{k+1}(\real^{n+1})\rightarrow S^{n}$を$(v_0,\cdots,v_{k})\mapsto v_{0}$とすると,$V_{k}(\real^{n})$をファイバーとするファイバー束となり,
  $\pi_i(V_k(\real^n))=0$ for $i=0,1,\cdots,n-k-1$である.
\end{prop}

\begin{proof}
  $U_i=\set{(x_0,\cdots,x_n)\in S^n}{x_i\neq 0}$とすると,$x=(x_0,\cdots,x_n)\in U_i$の直交補空間の正規直交基底$u_1,\cdots,u_n$を$x_0,\cdots,x_n$に関してなめらかにとることができる.行列$U(x)$を$U(x)=(u_1,\cdots,u_n)$とする.$\phi_i\colon U_i\times V_k(\real^n)\rightarrow V_{k+1}(\real^{n+1})|_{U_i}$を
  \[
  \phi_i(x,(v_1,\cdots,v_k)) = (x,U(x)v_1,\cdots,U(x)v_k)
  \]
  とすれば$\phi_i$は像への同相である.

  ホモトピー完全列
  \[
  \xymatrix{
    \cdots \ar[r] & \pi_{q+1}(S^n) \ar[r] & \pi_q(V_k(\real^n)) \ar[r] &
    \pi_q(V_{k+1}(\real^{n+1})) \ar[r] & \pi_q(S^n) \ar[r] & \cdots
  }
  \]
  において,$\pi_{q+1}(S^n) = 0,\:(q+1 < n)$であるから,
  \[
  \pi_q(V_{k+1}(\real^{n+1}))\simeq \pi_q(V_{k}(\real^{n}))
  \]
  よって$q<n-k$のとき
  \[
  \pi_q(V_k(\real^n))\simeq \pi_q(V_1(\real^{n-k+1}))\simeq \pi_q(S^{n-k})=0
  \]
\end{proof}

$V_k(\real^\infty)=\varinjlim V_k(\real^n)$, $\text{Gr}_k(\real^\infty)=\varinjlim \text{Gr}_k(\real^n)$とする.

\begin{prop}\label{universal O(k) bundle}
  $V_k(\real^\infty)\rightarrow \text{Gr}_k(\real^\infty)$はuniversal $O(k)$ bundleである.
\end{prop}

\begin{proof}
  $f\colon S^q\rightarrow V_k(\real^\infty)$を連続写像とする.$V_k(\real^\infty)$は各$V_k(\real^n)$を部分複体にもつようなCW複体の構造をもつ.$f(S^q)$はコンパクトであるから$f(S^q)\subset V_k(\real^n)$となる$n$が存在する\cite{hatcher Top}.十分大きく$n$をとれば$\pi_q(V_k(\real^n))=0$であるから$f$のホモトピー類も$0$である.よって定理\ref{steenrod}より$V_k(\real^\infty)\rightarrow \text{Gr}_k(\real^\infty)$はuniversal $O(k)$ bundleになる.
\end{proof}

\begin{theo}\label{universal bundle for subgroup}
  $G$をLie群,$H$を$G$の閉部分群とする.$EG\rightarrow EG/H$はuniversal $H$ bundleである.
\end{theo}

\begin{proof}
  $H$が閉部分群のとき$G\rightarrow G/H$は主$H$束になり,$EG\rightarrow EG/H$は局所的に$U\times G\rightarrow U\times(G/H)$の形をしている($U$は$BG$の開集合)から,$EG\rightarrow EG/H$は主$H$束である.$EG$は弱可縮であるから,定理\ref{steenrod}より$EG\rightarrow EG/H$はuniversal $H$ bundleである.
\end{proof}

任意のコンパクトLie群$G$は十分大きい$n$に対して$O(n)$に埋め込めることが知られている\cite{representation}.従って定理\ref{universal O(k) bundle}と定理\ref{universal bundle for subgroup}から,$V_n(\real^\infty)\rightarrow V_n(\real^\infty)/G$はuniversal $G$ bundleである.





\subsection{Borel構成}

\begin{defin}
  $G$が$X$に左から作用しているとき,$G$の$EG\times X$への左作用を
  \[
  g(x, e):=(gx, e\inv{g}) \quad \text{for } g\in G, x\in X, e\in EG 
  \]
  によって定める.$EG\times_GX:=(EG\times X)/G$としこれを$X$のhomotopy quotient という.このとき
  $H^*_G(X):=H^*(EG\times_GX)$を$X$の$G$同変コホモロジーという.
\end{defin}

\begin{eg}
  1点集合$\text{pt}$の$G$同変コホモロジーは
  \[
  EG\times_G\text{pt}=(EG\times \text{pt})/G\approx BG
  \]
  より
  \[
  H^*_G(\text{pt})\simeq H^*(BG)
  \]
  である。よって
  \[
  H^*_{S^1}(\text{pt})\simeq H^*(\mathbb{CP}^\infty)\simeq \integer[y]
  \]
\end{eg}


$X$の$G$同変コホモロジーは主$G$束$EG\rightarrow BG$の取り方に拠らないことを示そう.

$P\rightarrow B$を主$G$束とする.

写像$p\colon EG\times X\rightarrow EG\times_GX$と $p_X\colon EG\times_GX\rightarrow BG$を
\begin{align*}
  &p(x, e):=[x, e]\\
  &p_X([x, e]):=\pi(e)
\end{align*}
によって定める.

\begin{prop}
  \:
  \begin{enumerate}
    \item $p\colon EG\times X\rightarrow EG\times_GX$は主$G$束である
    \item $p_X\colon EG\times_GX\rightarrow BG$は$X$をファイバーとするファイバー束である
  \end{enumerate}
\end{prop}

\begin{proof}
  \:
  \begin{enumerate}
    \item $EG\rightarrow BG$は主$G$束であるので,局所的に$U\times G\rightarrow U$の形をしている($U$は$BG$の開集合).よって$EG\times X\rightarrow EG\times_G X$は局所的に$(U\times G)\times X\rightarrow (U\times G)\times_GX$である.$(U\times G)\times_GX\rightarrow U\times X$を
    \[
    [(u,g),x] \mapsto (u,gx)
    \]
    とすればこれは同相である.合成$(U\times G)\times X\rightarrow (U\times G)\times_GX\rightarrow U\times X$,$((u,g),x)\mapsto (u,gx)$は切断$U\times X\rightarrow (U\times G)\times X$,$(u,x)\mapsto((u,e),x)$を持つので自明束である.

    \item $EG\rightarrow BG$は局所的に$U\times G\rightarrow U$の形であり,$(U\times G)\times_GX\approx U\times X$であるから,$p_X$は局所的に$U\times X\rightarrow U$である.
  \end{enumerate}
\end{proof}

\begin{lemm}\label{borel const is weak hotmotopy equiv}
  $E$を弱可縮な$G$空間とし、$P\rightarrow B$を主$G$束とする。このとき$P\times_GE\rightarrow B$は弱ホモトピー同値である。
\end{lemm}

\begin{proof}
  $E$をファイバーとするファイバー束$p\colon P\times_GE\rightarrow B$のホモトピー完全列
  \[
  \xymatrix{
    \cdots \ar[r] & \pi_{q+1}(E) \ar[r] & \pi_{q+1}(P\times_GE) \ar[r]^{p_*} & \pi_{q+1}(B) \ar[r] & \pi_q(E) \ar[r] & \cdots
  }
  \]
  において,$E$は弱可縮であるから,$\pi_{q+1}(E)=\pi_{q}(E) = 0$.よって$p_*\colon\pi_q(P\times_GE)\rightarrow \pi_q(B)$は同型である.
\end{proof}

\begin{theo}\label{existance weak homotopy eq}
  $M$を$G$空間, $E\rightarrow B$, $E'\rightarrow B'$を主$G$束で$E,E'$はともに弱可縮であるとする。このとき弱ホモトピー同値$E\times_GM\rightarrow E'\times_GM$が存在する
\end{theo}

\begin{proof}
  
\end{proof}

\begin{theo}(Hatcher \cite{hatcher Top})\label{weak homotopy to cohomology}
  $X,Y$をCW複体とする.弱ホモトピー同値$f\colon X\rightarrow Y$は同型$f^*\colon H^*(Y)\rightarrow H^*(X)$を誘導する
\end{theo}

定理\ref{existance weak homotopy eq}と定理\ref{weak homotopy to cohomology}より, $EG\rightarrow BG$, $EG'\rightarrow BG'$がuniversal $G$-bundleであるとき、
\[
H^*(EG\times_GX)\simeq H^*(EG\times_GX')
\]
であることがわかる。

$X, Y$を$G$空間, $f\colon X\rightarrow Y$を$G$写像とする。$f_G\colon EG\times_GX\rightarrow EG\times_GY$を
\[
f_G([x, e])=[f(x), e]
\]
によって定めると$f_G$はwell-definedな連続写像となる。したがって$f_G$は同変コホモロジーの準同型
\[
f_G^*\colon H^*_G(Y)\rightarrow H^*_G(X)
\]
を誘導する。通常のコホモロジーの関手性と同様、同変コホモロジーも関手性をもつ
\begin{prop}\:
  \begin{enumerate}
    \item $(\id{X})_G^*=\id{H^*_G(X)}$
    \item $f\colon X\rightarrow Y$, $g\colon Y\rightarrow Z$に対して$(g\circ f)_G^*=(f_G^*)\circ(g_G^*)$
  \end{enumerate}
\end{prop}

任意の$G$空間$X$に対して、1点集合$\text{pt}$への自明な$G$写像は,準同型$H^*(BG)\simeq H^*_G(\text{pt})\rightarrow H^*_G(X)$を誘導する.したがって$H^*_G(X)$は$H^*(BG)$代数の構造を持つ.



\subsection{有限次元近似}

セクション1.1より,$G$がコンパクトLie群の場合,$G$のuniversal bundle $EG$に対して,有限次元多様体からなる主$G$束の族$EG_r\rightarrow BG_r, (r=1,2,\cdots)$であって$EG_r\subset EG_{r+1}$,$BG_r\subset BG_{r+1}$かつ$EG_{r+1}|_{BG_r} = EG_r$となるようなものが存在する.


$M$を非特異射影多様体とし$G$は$M$に左から作用しているとする.$X\subset M$を$G$不変な既約代数多様体で$\codim X =k$とする.$M_{G,r}=EG_r\times_GM$とすると$X_{G,r}\subset M_{G,r}$はコホモロジー類$[X_{G,r}]\in H^k(M_{G,r})$を定める\cite{fulton young tableaux}.$[X_{G,r+1}]|_{M_{G,r}} = [X_{G,r}]$が成り立つから,その極限$[X]\in H^k_G(M)$が定まる.

$X^{sm}$を$X$の非特異点のなす部分多様体とする.

\begin{prop}\label{restriction to fixed point}
  $G=(S^1)^n$とし,$p$を$X$の非特異な点で$G$固定点とする.$[X]|_p\in H^k_G(p)$は$X^{sm}\subset M$の法束の$p$におけるウェイトの積に等しい.
\end{prop}

\begin{proof}
  $[X]|_p$を計算するために,その有限次元近似$[X_{G,r}]|_{p_{G,r}}\in H^k(p_{G,r}=BG_r)$を計算する.$[X_{G,r}]|_{p_{G,r}}$は$X^{sm}_{G,r}\subset M_{G,r}$の法束$\mathcal{N}_{G,r}$を$p_{G,r}$に制限したベクトル束$\mathcal{N}_{G,r}|_{p_{G,r}}$のEuler類に他ならない.

  よって$G$の表現$V$に対して,ベクトル束$V_{G,r}\rightarrow p_{G,r}=BG_r$のEuler類を計算する.簡単のため,$n=1$,$V=\complex$で$G$の$V$への作用が
  \[
  t\cdot v := t^mv \quad \text{for }t\in G,v\in\complex
  \]
  となっているとする.このとき,$V_{G,r} = S^{2r+1}\times_G V$,$BG_r = \mathbb{CP}^r$である.$U_i = \set{[z_0:\cdots:z_r]\in\mathbb{CP}^r}{z_i\neq 0}$とし,$\phi_i:U_i\times \complex\rightarrow V_{G,r}|_{U_i}$を
  \[
  \phi_i([z_0:\cdots:z_i:\cdots:z_r],v) = \left[\left(\frac{z_0|z_i|}{z_iN},\cdots,\frac{|z_i|}{N},\cdots,\frac{z_r|z_i|}{z_iN}\right), v \right]
  \]
  とする.ここで$N=\sqrt{|z_0|^2+\cdots+|z_i|^2+\cdots+|z_r|^2}$である.$\phi_i$は同相であり,変換関数$\inv{\phi_i}\phi_j:U_i\cap U_j\rightarrow \complex^\times$を計算すると,
  \begin{align*}
    \inv{\phi_j}\phi_i([z_0:\cdots:z_r],v)
    &= \inv{\phi_j}\left[\left(\frac{z_0|z_i|}{z_iN},\cdots,\frac{|z_i|}{N},\cdots,\frac{z_j|z_i|}{z_iN},\cdots\frac{z_r|z_i|}{z_iN}\right), v \right]\\
    &=\inv{\phi_j}\left[\left(\frac{z_0|z_j|}{z_jN},\cdots,\frac{z_i|z_j|}{z_jN},\cdots,\frac{|z_j|}{N},\cdots\frac{z_r|z_j|}{z_jN}\right)\cdot \frac{z_j|z_i|}{|z_j|z_i},v\right]\\
    &=\inv{\phi_j}\left[\left(\frac{z_0|z_j|}{z_jN},\cdots,\frac{z_i|z_j|}{z_jN},\cdots,\frac{|z_j|}{N},\cdots\frac{z_r|z_j|}{z_jN}\right) , \left(\frac{z_i|z_j|}{|z_i|z_j}\right)^mv\right]\\
    &=\left([z_0:\cdots:z_r],\left(\frac{z_i|z_j|}{|z_i|z_j}\right)^mv\right)
  \end{align*}
  したがって$\tau$を$\mathbb{CP}^r$のtautological bundleとすると,$V_{G,r}\simeq \tau^{\otimes m}$であることがわかる.よって$e(\tau)$を$\tau$のEuler類とすると,
  \[
  e(V_{G,r}) = m e(\tau)
  \]

  
\end{proof}






\subsection{Weil model/Cartan model}\label{weil/cartan}

\begin{defin}
  $\mathfrak{g}$を$\real$上のLie代数とする.可換な次数付き$\real$代数$A=\bigoplus_{k=0}^\infty A^k$と線形写像
  \begin{align*}
    &d\colon A^k\rightarrow A^{k+1}\\
    &\iota_X\colon A^k\rightarrow A^{k-1},\quad \forall X\in\mathfrak{g}\\
    &\mathcal{L}_X\colon A^k\rightarrow A^{k},\quad \text\forall X\in\mathfrak{g}
  \end{align*}
  の組$(A,d,\iota,\mathcal{L})$が$\mathfrak{g}$-differential graded algebraであるとは,
  \begin{enumerate}
    \item $d^2=0$かつ$d$は反微分.すなわち
    $
    d(ab) = (da)b + (-1)^{\deg a}a (db)
    $
    を満たす.
    \item $\iota_A^2=0$かつ$\iota_A$は反微分.
    \item $\mathcal{L}_A$はCartanのhomotopy formulaを満たす.すなわち$\mathcal{L}_A = \delta\circ \iota_A + \iota_A + \delta$.
  \end{enumerate}
  が成り立つことをいう.ここで$A$が可換であるとは,
  $
  ab = (-1)^{(\deg a)(\deg b)}ba
  $
  が成り立つことをいう.
\end{defin}

\begin{eg}
  $G$を連結なコンパクトLie群,$\mathfrak{g}$を$G$のLie代数,$M$を$G$が作用するなめらかな多様体とする.$\Omega(M)=\bigoplus_{k=0}^\infty\Omega^k(M)$を$M$のde Rham複体,$d\colon\Omega^k(M)\rightarrow\Omega^{k+1}(M)$を外微分とする.

  $A\in\mathfrak{g}$に対して,$\iota_A\colon\Omega^k(M)\rightarrow \Omega^{k-1}(M)$を
  \[
  \iota_A\omega(X_1,\cdots,X_{k-1}) = \omega(\underline{A},X_1,\cdots,X_{k-1})
  \]
  とする.ここで$\underline{A}$は$A$の基本ベクトル場である.すなわち,$p\in M$に対して
  \[
  \underline{A}_p = \left.\frac{d}{dt}\exp(-At)p\right|_{t=0}
  \]
  で定まるベクトル場である.

  $\mathcal{L}_A := d\circ\iota_A + \iota_A\circ d$として,$(\Omega(M),d,\iota,\mathcal{L})$は$\mathfrak{g}$-differential graded algebraになる.
\end{eg}

\begin{eg}
$\mathfrak{g}$をLie代数とする.$\mathfrak{g}$の基底$X_1,\cdots,X_n$を固定し,その双対基底を$\alpha_1,\cdots,\alpha_n$とおく.$\bigwedge(\mathfrak{g}^\vee)=\bigoplus_{k=0}^{n}\bigwedge^k(\mathfrak{g}^\vee)$を$\mathfrak{g}^\vee$の交代テンソル空間,$S(\mathfrak{g})=\bigoplus_{k=0}^\infty S^k(\mathfrak{g}^\vee)$を$\mathfrak{g}^\vee$の対称テンソル空間とする.$W(\mathfrak{g})=\bigwedge(\mathfrak{g}^\vee)\otimes S(\mathfrak{g}^\vee)$とし,
\begin{align*}
  &\theta_i = \alpha_i\otimes 1 \in\bigwedge(\mathfrak{g}^\vee)\otimes S(\mathfrak{g}^\vee)\\
  &u_i = 1\otimes\alpha_i \in\bigwedge(\mathfrak{g}^\vee)\otimes S(\mathfrak{g}^\vee)
\end{align*}
とおく.積の記号$\otimes$を省略して,$\theta_i$を$\alpha_i\in\bigwedge(\mathfrak{g}^\vee)$と同一視し,$u_i$を$\alpha_i\in S(\mathfrak{g}^\vee)$と同一視する.$\bigwedge(\mathfrak{g}^\vee)$は$\theta_1,\cdots,\theta_n$で生成される交代テンソル代数$\bigwedge(\theta_1,\cdots,\theta_n)$に同型であり,$S(\mathfrak{g}^\vee)$は$u_1,\cdots,u_n$で生成される多項式環$\real[u_1,\cdots,u_n]$に同型である.$\theta_i$の次数は$1$,$u_i$の次数は$2$であるから,$W(\mathfrak{g})$は次のような次数付き代数の構造をもつ.
\[
W(\mathfrak{g})= \bigoplus_{k=0}^\infty W^k(\mathfrak{g}) \simeq\bigoplus_{k=0}^\infty\bigoplus_{p+2q=k}\bigwedge^p(\theta_1,\cdots,\theta_n)\otimes S^q(u_1,\cdots,u_n)
\]

$c^k_{ij}$を$\mathfrak{g}$の$X_1,\cdots,X_n$に関する構造定数とする.すなわち,$[X_i,X_j]=\sum_{k=1}^nc^k_{ij}X_k$とする.線形写像$\delta\colon W^m(\mathfrak{g})\rightarrow W^{m+1}(\mathfrak{g})$を
\[
\delta\theta_k := u_k - \frac{1}{2}\sum_{i,j}c^k_{ij}\theta_i\theta_j,\quad \delta u_k := \sum_{i,j}c^k_{ij}u_i\theta_j
\]
として,$p+2q=m$なる$p,q$に対して
\begin{align*}
  \delta(\theta_{i_1}\cdots\theta_{i_p}\cdot u_{j_1}\cdots u_{j_q}) := \sum_{x=1}^p(-1)^{x-1}\theta_{i_1}&\cdots(\delta\theta_{i_x})\cdots\theta_{i_p}\cdot u_{j_1}\cdots u_{j_q} \notag \\ 
  &+ (-1)^p\sum_{y=1}^q\theta_{i_1}\cdots\theta_{i_p}\cdot u_{j_1}\cdots(\delta u_{j_y})\cdots u_{j_q}
\end{align*}
によって定義する.

次に$A\in\mathfrak{g}$に対して,線形写像$\iota_A\colon W^m(\mathfrak{g})\rightarrow W^{m-1}(\mathfrak{g})$を
\[
\iota_A(\theta_i):=\alpha_i(A),\quad \iota_A(u_i)=0
\]
として,$p+2q=m$なる$p,q$に対して
\begin{align*}
  \iota_A(\theta_{i_1}\cdots\theta_{i_p}\cdot u_{j_1}\cdots u_{j_q}) := \sum_{x=1}^p(-1)^{x-1}\theta_{i_1}&\cdots(\iota_A\theta_{i_x})\cdots\theta_{i_p}\cdot u_{j_1}\cdots u_{j_q} \notag \\ 
  &+ (-1)^p\sum_{y=1}^q\theta_{i_1}\cdots\theta_{i_p}\cdot u_{j_1}\cdots(\iota_A u_{j_y})\cdots u_{j_q}
\end{align*}
によって定義する.

最後に$A\in\mathfrak{g}$に対して$\mathcal{L}_A\colon W^m(\mathfrak{g})\rightarrow W^m(\mathfrak{g})$を
$
\mathcal{L}_A:=\delta\circ\iota_A + \iota_A\circ\delta
$
として,$(W(\mathfrak{g}),\delta,\iota,\mathcal{L})$は$\mathfrak{g}$-differential graded algebraになる.
\end{eg}

\begin{eg}\label{tonsor of g-dga}
  $(A_1,d_1,\iota_1,\mathcal{L}_1),(A_2,d_2,\iota_2,\mathcal{L}_2)$を$\mathfrak{g}$-differential graded algebraとする.$A_1\otimes A_2$に
  \[
  (a_1\otimes a_2)\cdot (b_1\otimes b_2):=(-1)^{(\deg a_2)(\deg b_1)}a_1b_1\otimes a_2b_2
  \]
  によって積を入れ,次数を$\deg(a_1\otimes a_2) := \deg a_1 + \deg a_2$とすると$A_1\otimes A_2$は可換な次数付き代数の構造をもつ.
  $d(a_1\otimes a_2) := d_1a_1\otimes a_2 + (-1)^{\deg a_1}a_1\otimes d_2a_2$,$\iota_X(a_1\otimes a_2) := \iota_{1,X}a_1\otimes a_2 + (-1)^{\deg a_1}a_1\otimes\iota_{2,X}a_2$,$\mathcal{L}_X := d\circ \iota_X + \iota_X \circ d$によってそれぞれを定義すれば,$(A_1\otimes A_2,d,\iota,\mathcal{L})$は$\mathfrak{g}$-differential graded algbraとなる.
\end{eg}

\begin{defin}
  $(A,d,\iota,\mathcal{L})$を$\mathfrak{g}$-differential graded algebraとする.$\alpha\in A$がbasicであるとは,
  \[
  \iota_X(\alpha) = 0,\quad\mathcal{L}_X(\alpha) = 0,\quad \forall X\in\mathfrak{g}
  \]
  が成り立つことをいう.$A_{bas}=\set{\alpha\in A}{\alpha\text{ is basic}}$とおき,$d,\iota,\mathcal{L}$をすべて$A_{bas}$に制限すると,$(A_{bas},d,\iota,\mathcal{L})$は$\mathfrak{g}$-differential graded algebraになる.
\end{defin}

\begin{defin}
  $M$を$G$が作用するなめらかな多様体とする.$\mathfrak{g}$-differential graded algebra $(W(\mathfrak{g})\otimes\Omega(M))_{bas}$をWeil modelと呼ぶ.
\end{defin}

\begin{theo}(Equivariant de Rhamの定理\cite{tu equivariant})
  $G$が連結なコンパクトLie群の場合,次数付き代数の同型
  \[
  H^*_G(M;\real)\simeq H^*(\{(W(\mathfrak{g})\otimes\Omega(M))_{bas},d\})
  \]
  が存在する.
\end{theo}

\begin{eg}
  $G=S^1$とする.$\mathfrak{g}$の基底$X$を固定すると$\mathfrak{g}=\real X$であり,構造定数$c^k_{ij}$はすべて$0$である.よって
  \begin{align*}
    &d\theta = u,\quad du = 0\\
    &\iota_X\theta = 1,\quad\iota_Xu = 0\\
    &\mathcal{L}_X\theta = 0,\quad\mathcal{L}_Xu=0
  \end{align*}
  である.
  \[
  W(\mathfrak{g})\simeq \bigwedge(\theta)\otimes S(u) \simeq \real[u]\oplus \real[u]\theta
  \]
  より,
  \[
  W(\mathfrak{g})\otimes\Omega(M)\simeq \Omega(M)[u]\oplus\Omega(M)[u]\theta
  \]
  である.$\alpha\in W(\mathfrak{g})$を$\alpha = f(u) + \theta g(u)$とおく.ただし$f(u),g(u)\in\Omega(M)[u]$($\Omega(M)$を係数とする$u$の多項式環)である.
  \begin{align*}
    \iota_X(\alpha) = \iota_Xf(u) + g(u) - \theta(\iota_Xg(u))
  \end{align*}
  であるから,$\iota_X\alpha = 0$ならば,
  \[
  g(u) = -\iota_Xf(u),\quad \iota_Xg(u) = 0
  \]
  である.逆に$g(u) = -\iota_Xf(u)$なら自動的に$\iota_Xg(u)=0$であるので,
  \[
  \iota_X\alpha = 0\Leftrightarrow g(u) = -\iota_Xf(u)
  \]
  また,
  \[
  \mathcal{L}_X\alpha = \mathcal{L}_Xf(u) + \theta\mathcal{L}_Xg(u)
  \]
  より,$\mathcal{L}_X\alpha = 0$ならば
  \[
  \mathcal{L}_Xf(u) = 0,\quad \mathcal{L}_Xg(u)=0
  \]
  である.ここで,$\iota_X\alpha = 0$であるとき,$g(u) = -\iota_Xf(u)$だから,
  \[
  \mathcal{L}_Xg(u) = -\iota_X\mathcal{L}_Xf(u)=0
  \]
  したがって,
  \[
  \alpha\text{ is basic }\Leftrightarrow g(u) = -\iota_Xf(u),\quad \mathcal{L}_Xf(u) = 0
  \]
  そこで,$\Omega(M)^G$を$\mathcal{L}_X\eta = 0$を満たす微分形式のなす部分複体とすれば,
  \[
  (W(\mathfrak{g})\otimes\Omega(M))_{bas} = \set{(1-\theta\iota_X)f(u)}{f(u)\in\Omega(M)^G[u]}
  \]
  である.とくに$F\colon (W(\mathfrak{g})\otimes\Omega(M))_{bas}\rightarrow \Omega(M)^G[u]$を$F((1-\theta\iota_X)f(u)) = f(u)$とすれば,$F$は次数付き代数の同型になる\cite{tu equivariant}.
  このとき,
  \begin{align*}
    F\circ d\circ \inv{F}(f(u))
    &=F\circ d(f(u) - \theta\iota_Xf(u))\\
    &=F(df(u) - u\iota_Xf(u) + \theta d\iota_Xf(u))\\
    &=F(df(u) - u\iota_Xf(u)-\theta\iota_Xdf(u))\\
    &=F((1-\theta\iota_X)(df(u) - u\iota_Xf(u)))\\
    &=(d-u\iota_X)f(u)
  \end{align*}
  であるから,$d_X\colon\Omega^G(M)[u]\rightarrow \Omega^G(M)[u]$を
  \[
  d_X(f(u)) = (d-u\iota_X)f(u)
  \]
  $(\Omega^G(M)[u],d_X)$はdifferential graded algebraとして$(W(\mathfrak{g})\otimes\Omega(M))_{bas}$と同型である.これをCartan modelと呼び,$f(u)\in\Omega^G(M)[u]$をequvariant differential formと呼ぶ.
\end{eg}

\begin{eg}
  同様に$G=(S^1)^n$の場合,$d_{X_1,\cdots,X_n}\colon\Omega^G(M)[u_1,\cdots,u_n]\rightarrow\Omega^G(M)[u]$を
  \[
  d_{X_1,\cdots,X_n}f(u_1,\cdots,u_n) =(d-\sum_{i}u_i\iota_{X_i})f(u_1,\cdots,u_n) 
  \]
  とし,
  $
  F\colon (W(\mathfrak{g})\otimes\Omega(M))_{bas}\rightarrow \Omega^G(M)[u_1,\cdots,u_n]
  $
  を
  \[
  F(f(u_1,\cdots,u_n) + \sum_{I}\theta_If(u)) = f(u),\quad I=(i_1<\cdots<i_p),\theta_I = \theta_{i_1}\cdots\theta_{i_p}
  \]
  とすれば$F$はdifferential graded algebraの同型となる.
\end{eg}

\begin{eg}
  $G=(S^1)^n$,$M=\text{pt}$とすると,$\Omega(M)\simeq\real$,$d_{X_1,\cdots,X_n} = 0$であるから,$H^*_G(M)\simeq\Omega^G(M)[u_1,\cdots,u_n]\simeq \real[u_1,\cdots,u_n]$
\end{eg}




\subsection{localization theorem}


$G=(S^1)^n$とし,$M$を$G$が作用する向き付けられたコンパクト多様体とする.

\begin{defin}
  $\alpha\in\Omega^G(M)[u_1,\cdots,u_n]$に対して,$\int_M\alpha\colon\mathfrak{g}\rightarrow\real$を
  \[
  \left(\int_M\alpha\right)(X) := \int_M\alpha(X)
  \]
  によって定義する.
\end{defin}

\begin{theo}[Atiyah-Bott]
  $M$の固定点の集合$M^G$が有限であるとする.$G$の表現$T_pM$
  \[
  \int_M\alpha = \sum_{p\in M^G}\frac{\alpha|_p}{\prod_{}}
  \]
\end{theo}

