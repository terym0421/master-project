\section{同変コホモロジー}
\subsection{universal bundle}

$G$をコンパクトLie群とする.以下考える位相空間はすべてCW複体であるとする.本論文では特に断らない限りコホモロジーの係数はすべて$\integer$である.

\begin{defin}
  次の2つの条件を満たす主$G$束$f\colon EG\rightarrow BG$をuniversal $G$ bundleという.
  \begin{enumerate}
    \item 任意の主$G$束$h\colon E\rightarrow B$に対して連続写像$h\colon B\rightarrow EB$が存在して$E\simeq h^*EG$が成り立つ.
    \item $h_0,h_1\colon B\rightarrow EB$に対して$h_0^*EG\simeq h_1^*EG$ならば$h_0,h_1$はホモトピックである.
  \end{enumerate}
\end{defin}

\begin{theo}(Steenrod \cite{steenrod})\label{steenrod}
  主$G$束$E\rightarrow B$がuniversal $G$ bundleであるための必要十分条件は$E$が弱可縮であることである.
\end{theo}

\begin{eg}
  $S^\infty$\footnote{入る位相について明記?}は可縮であるので,$T=S^1$に対して$S^\infty\rightarrow \mathbb{CP}^\infty$はuniversal $T$ bundleである.
\end{eg}

$G$がコンパクトLie群の場合,universal $G$ bundleは直交群$O(k)$のuniversal bundleから構成できる.

$V_k(\real^n)$を$\real^n$の$k$個の正規直交なベクトル$(v_1,\cdots,v_k)$のなす空間とし,$\text{Gr}_k(\real^n)$を$\real^n$の$k$次元部分空間全体のなす空間とする.$V_k(\real^n)$をStiefel多様体,$\text{Gr}_k(\real^n)$をGrassmann多様体という.$(v_1,\cdots,v_k)\in V_k(\real^n)$を$n\times k$行列とみなすことで,$O(k)$は$V_k(\real^n)$に自由に右作用する.また$(v_1,\cdots,v_k)\in V_k(\real^n)$に対して,$v_1,\cdots,v_k$の生成する部分空間を対応させることで写像$V_k(\real^n)\rightarrow \text{Gr}_k(\real^n)$が定まる.

\begin{prop}
  $V_k(\real^n)\rightarrow \text{Gr}_k(\real^n)$は主$O(k)$束である.
\end{prop}

\begin{proof}
  簡単のため,$k=2,n=4$の場合に示す.$U_{1,2}\subset \text{Gr}_2(\real^4)$を
  \begin{equation}\label{basis}
    \left(\begin{array}{c}
      1\\0\\ * \\ *
    \end{array}\right),\quad \left(\begin{array}{c}
      0\\1\\ * \\ *
    \end{array}\right)
  \end{equation}
  の形のベクトルで生成される部分空間のなす集合とする.ここで$*$は任意の実数である.(\ref{basis})にSchmidtの直交化法を施して得られるベクトルを$v_1,v_2$とし,(\ref{basis})に$(v_1,v_2)$を対応させれば,切断$s_{1,2}\colon U_{1,2}\rightarrow V_k(\real^n)$
  得られる.$\phi_{1,2}\colon U_{1,2}\times O(2)\rightarrow V_{2}(\real^4)|_{U_{1,2}}$を
  \[
  \phi_{1,2}(\left\langle
    \left(\begin{array}{c}
      1\\0\\ * \\ *
    \end{array}\right),\:\left(\begin{array}{c}
      0\\1\\ * \\ *
    \end{array}\right)
  \right\rangle, P) = (v_1,v_2)P
  \]
  とすれば$\phi_{1,2}$は像への同相であり,$O(2)$の右作用と可換である.
  
  同様に$U_{i,j},s_{i,j},\phi_{i,j}$を$1\leq i<j\leq 4$に対して定義すれば$G$の作用と可換な局所自明化が得られる.
\end{proof}

\begin{prop}
  $V_{k+1}(\real^{n+1})\rightarrow S^{n}$を$(v_0,\cdots,v_{k})\mapsto v_{0}$とすると,$V_{k}(\real^{n})$をファイバーとするファイバー束となり,
  $\pi_i(V_k(\real^n))=0$ for $i=0,1,\cdots,n-k-1$である.
\end{prop}

\begin{proof}
  $U_i=\set{(x_0,\cdots,x_n)\in S^n}{x_i\neq 0}$とすると,$x=(x_0,\cdots,x_n)\in U_i$の直交補空間の正規直交基底$u_1,\cdots,u_n$を$x_0,\cdots,x_n$に関してなめらかにとることができる.行列$U(x)$を$U(x)=(u_1,\cdots,u_n)$とする.$\phi_i\colon U_i\times V_k(\real^n)\rightarrow V_{k+1}(\real^{n+1})|_{U_i}$を
  \[
  \phi_i(x,(v_1,\cdots,v_k)) = (x,U(x)v_1,\cdots,U(x)v_k)
  \]
  とすれば$\phi_i$は像への同相である.

  ホモトピー完全列
  \[
  \xymatrix{
    \cdots \ar[r] & \pi_{q+1}(S^n) \ar[r] & \pi_q(V_k(\real^n)) \ar[r] &
    \pi_q(V_{k+1}(\real^{n+1})) \ar[r] & \pi_q(S^n) \ar[r] & \cdots
  }
  \]
  において,$\pi_{q+1}(S^n) = 0,\:(q+1 < n)$であるから,
  \[
  \pi_q(V_{k+1}(\real^{n+1}))\simeq \pi_q(V_{k}(\real^{n}))
  \]
  よって$q<n-k$のとき
  \[
  \pi_q(V_k(\real^n))\simeq \pi_q(V_1(\real^{n-k+1}))\simeq \pi_q(S^{n-k})=0
  \]
\end{proof}

$V_k(\real^\infty)=\varinjlim V_k(\real^n)$, $\text{Gr}_k(\real^\infty)=\varinjlim \text{Gr}_k(\real^n)$とする.

\begin{prop}\label{universal O(k) bundle}
  $V_k(\real^\infty)\rightarrow \text{Gr}_k(\real^\infty)$はuniversal $O(k)$ bundleである.
\end{prop}

\begin{proof}
  $f\colon S^q\rightarrow V_k(\real^\infty)$を連続写像とする.$V_k(\real^\infty)$は各$V_k(\real^n)$を部分複体にもつようなCW複体の構造をもつ.$f(S^q)$はコンパクトであるから$f(S^q)\subset V_k(\real^n)$となる$n$が存在する\cite{hatcher Top}.十分大きく$n$をとれば$\pi_q(V_k(\real^n))=0$であるから$f$のホモトピー類も$0$である.
\end{proof}

\begin{theo}\label{universal bundle for subgroup}
  $G$をLie群,$H$を$G$の閉部分群とする.$EG\rightarrow EG/H$はuniversal $H$ bundleである.
\end{theo}

\begin{proof}
  $H$が閉部分群のとき$G\rightarrow G/H$は主$H$束になり,$EG\rightarrow EG/H$は局所的に$U\times G\rightarrow U\times(G/H)$の形をしている($U$は$BG$の開集合)から,$EG\rightarrow EG/H$は主$H$束である.$EG$は弱可縮であるから,定理\ref{steenrod}より$EG\rightarrow EG/H$はuniversal $H$ bundleである.
\end{proof}

任意のコンパクトLie群$G$は十分大きい$n$に対して$O(n)$に埋め込めることが知られている\cite{representation}.従って定理\ref{universal O(k) bundle}と定理\ref{universal bundle for subgroup}から,$V_n(\real^\infty)\rightarrow V_n(\real^\infty)/G$はuniversal $G$ bundleである.





\subsection{Borel構成}

\begin{defin}
  $G$が$X$に左から作用しているとき,$G$の$EG\times X$への左作用を
  \[
  g(x, e):=(gx, e\inv{g}) \quad \text{for } g\in G, x\in X, e\in EG 
  \]
  によって定める.$EG\times_GX:=(EG\times X)/G$としこれを$X$のhomotopy quotient という.このとき
  $H^*_G(X):=H^*(EG\times_GX)$を$X$の$G$同変コホモロジーという.
\end{defin}

\begin{eg}
  1点集合$\text{pt}$の$G$同変コホモロジーは
  \[
  EG\times_G\text{pt}=(EG\times \text{pt})/G\approx BG
  \]
  より
  \[
  H^*_G(\text{pt})\simeq H^*(BG)
  \]
  である。よって
  \[
  H^*_{\complex^*}(\text{pt})\simeq H^*(\mathbb{CP}^\infty)\simeq \integer[y]
  \]
\end{eg}


$X$の$G$同変コホモロジーは主$G$束$EG\rightarrow BG$の取り方に拠らないことを示そう.
写像$p\colon EG\times X\rightarrow EG\times_GX$と $p_X\colon EG\times_GX\rightarrow BG$を
\begin{align*}
  &p(x, e):=[x, e]\\
  &p_X([x, e]):=\pi(e)
\end{align*}
によって定める.

\begin{prop}
  \:
  \begin{enumerate}
    \item $p\colon EG\times X\rightarrow EG\times_GX$は主$G$束である
    \item $p_X\colon EG\times_GX\rightarrow BG$は$X$をファイバーとするファイバー束である
  \end{enumerate}
\end{prop}

\begin{proof}
  \:
  \begin{enumerate}
    \item $EG\rightarrow BG$は主$G$束であるので,局所的に$U\times G\rightarrow U$の形をしている($U$は$BG$の開集合).よって$EG\times X\rightarrow EG\times_G X$は局所的に$(U\times G)\times X\rightarrow (U\times G)\times_GX$である.$(U\times G)\times_GX\rightarrow U\times X$を
    \[
    [(u,g),x] \mapsto (u,gx)
    \]
    とすればこれは同相である.合成$(U\times G)\times X\rightarrow U\times X$, $((u,g),x)\mapsto (u,gx)$は自明束である.実際
  \end{enumerate}
\end{proof}

連続写像$f\colon X\rightarrow Y$がホモトピー群の同型
\[
f_*\colon\pi_q(X, x)\rightarrow \pi_q(Y, f(x))\quad\text{for } x\in X, q>0 
\]
を誘導するとき、$f$を弱ホモトピー同値\footnote{弱可縮よりも先に書く?}という。

\begin{lemm}
  $E$を弱可縮な$G$空間とし、$P\rightarrow B$を主$G$束とする。このとき$(E\times P)/G\rightarrow B$は弱ホモトピー同値である。
\end{lemm}

\begin{proof}
  
\end{proof}

\begin{theo}\label{existance weak homotopy eq}
  $M$を$G$空間, $E\rightarrow B$, $E'\rightarrow B'$を主$G$束で$E,E'$はともに弱可縮であるとする。このとき弱ホモトピー同値$E\times_GM\rightarrow E'\times_GM$が存在する
\end{theo}

\begin{proof}
  
\end{proof}

\begin{theo}(Hatcher \cite{hatcher Top})\label{weak homotopy to cohomology}
  弱ホモトピー同値$f\colon X\rightarrow Y$は同型$f^*\colon H^*(Y)\rightarrow H^*(X)$を誘導する
\end{theo}

定理\ref{existance weak homotopy eq}と定理\ref{weak homotopy to cohomology}より, $EG\rightarrow BG$, $EG'\rightarrow BG'$がuniversal $G$-bundleであるとき、
\[
H^*(EG\times_GX)\simeq H^*(EG\times_GX')
\]
であることがわかる。

$X, Y$を$G$空間, $f\colon X\rightarrow Y$を$G$写像とする。$f_G\colon EG\times_GX\rightarrow EG\times_GY$を
\[
f_G([x, e])=[f(x), e]
\]
によって定めると$f_G$はwell-definedな連続写像となる。したがって$f_G$は同変コホモロジーの準同型
\[
f_G^*\colon H^*_G(Y)\rightarrow H^*_G(X)
\]
を誘導する。通常のコホモロジーの関手性と同様、同変コホモロジーも関手性をもつ
\begin{prop}\:
  \begin{enumerate}
    \item $(\id{X})_G^*=\id{H^*_G(X)}$
    \item $f\colon X\rightarrow Y$, $g\colon Y\rightarrow Z$に対して$(g\circ f)_G^*=(f_G^*)\circ(g_G^*)$
  \end{enumerate}
\end{prop}

任意の$G$空間$X$に対して、1点集合$\text{pt}$への自明な$G$写像は,準同型$H^*(BG)\simeq H^*_G(\text{pt})\rightarrow H^*_G(X)$を誘導するから、$H^*_G(X)$は$H^*(BG)$代数の構造を持つ.



\subsection{同変コホモロジーの有限次元近似}

セクション1.1より,$G$がコンパクトLie群の場合,$G$のuniversal bundle $EG$に対して,有限次元多様体からなる主$G$束の族$EG_r\rightarrow BG_r, (r=1,2,\cdots)$であって$EG_r\subset EG_{r+1}$,$BG_r\subset BG_{r+1}$かつ$EG_{r+1}|_{BG_r} = EG_r$となるようなものが存在する.


$M$を非特異射影多様体とし$G$は$M$に左から作用しているとする.$X\subset M$を$G$不変な既約代数多様体で$\codim X =k$とする.$M_{G,r}=EG_r\times_GM$とすると$X_{G,r}\subset M_{G,r}$はコホモロジー類$[X_{G,r}]\in H^k(M_{G,r})$を定める\cite{fulton young tableaux}.$[X_{G,r+1}]|_{M_{G,r}} = [X_{G,r}]$が成り立つから,その極限$[X]\in H^k_G(M)$が定まる.


\begin{prop}\label{restriction to fixed point}
  $G=(S^1)^n$とし,$p\in X^{sm}$を$G$固定点とする.$[X]|_p\in H^k_G(p)$は$X\subset M$の法束の$p$におけるウェイトの積に等しい.
\end{prop}

\begin{proof}
  $[X]|_p$を計算するために,その有限次元近似$[X_{G,r}]|_{p_{G,r}}\in H^k(p_{G,r}=BG_r)$を計算する.$[X_{G,r}]|_{p_{G,r}}$は$X_{G,r}\subset M_{G,r}$の法束$\mathcal{N}_{G,r}$を$p_{G,r}$に制限したベクトル束$\mathcal{N}_{G,r}|_{p_{G,r}}$のEuler類に他ならない.

  よって$G$の表現$V$に対して,ベクトル束$V_{G,r}\rightarrow p_{G,r}=BG_r$のEuler類を計算する.簡単のため,$n=1$,$V=\complex$で$G$の$V$への作用が
  \[
  t\cdot v := t^mv \quad \text{for }t\in G,v\in\complex
  \]
  となっているとする.このとき,$V_{G,r} = S^{2r+1}\times_G V$,$BG_r = \mathbb{CP}^r$である.$U_i = \set{[z_0:\cdots:z_r]\in\mathbb{CP}^r}{z_i\neq 0}$とし,$\phi_i:U_i\times \complex\rightarrow V_{G,r}|_{U_i}$を
  \[
  \phi_i([z_0:\cdots:z_i:\cdots:z_r],v) = \left[\left(\frac{z_0|z_i|}{z_iN},\cdots,\frac{|z_i|}{N},\cdots,\frac{z_r|z_i|}{z_iN}\right), v \right]
  \]
  とする.ここで$N=\sqrt{|z_0|^2+\cdots+|z_i|^2+\cdots+|z_r|^2}$である.$\phi_i$は同相であり,変換関数$\inv{\phi_i}\phi_j:U_i\cap U_j\rightarrow \complex^\times$を計算すると,
  \begin{align*}
    \inv{\phi_j}\phi_i([z_0:\cdots:z_r],v)
    &= \inv{\phi_j}\left[\left(\frac{z_0|z_i|}{z_iN},\cdots,\frac{|z_i|}{N},\cdots,\frac{z_j|z_i|}{z_iN},\cdots\frac{z_r|z_i|}{z_iN}\right), v \right]\\
    &=\inv{\phi_j}\left[\left(\frac{z_0|z_j|}{z_jN},\cdots,\frac{z_i|z_j|}{z_jN},\cdots,\frac{|z_j|}{N},\cdots\frac{z_r|z_j|}{z_jN}\right)\cdot \frac{z_j|z_i|}{|z_j|z_i},v\right]\\
    &=\inv{\phi_j}\left[\left(\frac{z_0|z_j|}{z_jN},\cdots,\frac{z_i|z_j|}{z_jN},\cdots,\frac{|z_j|}{N},\cdots\frac{z_r|z_j|}{z_jN}\right) , \left(\frac{z_i|z_j|}{|z_i|z_j}\right)^mv\right]\\
    &=\left([z_0:\cdots:z_r],\left(\frac{z_i|z_j|}{|z_i|z_j}\right)^mv\right)
  \end{align*}
  したがって$\tau$を$\mathbb{CP}^r$のtautological bundleとすると,$V_{G,r}\simeq \tau^{\otimes m}$であることがわかる.よって
  \[
  e(V_{G,r}) = m e(\tau)
  \]
\end{proof}


\subsection{Weil/Cartanモデル}





\subsection{localization theorem}

