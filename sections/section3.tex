\section{同変Schubert計算}
\subsection{Grassmann多様体の同変コホモロジー}
$\text{Gr}_k(\complex^n)=\set{V\subset \complex^n}{\dim V = k}$をGrassmann多様体という。$T=\complex^n$とするとき、$T$は$\complex^n$に
\[
(t_1,\cdots,t_n)\cdot(x_1,\cdots,x_n)=(t_1x_1,\cdots,t_nx_n)
\]
によって左から作用する。この作用は自然に$\text{Gr}_k(\complex^n)$への作用を誘導し、$\text{Gr}_k(\complex^n)$は$T$空間となる。$\text{Gr}_k(\complex^n)$の$T$同変コホモロジーの構造は組み合わせ的に決定することができる。
$\binom{n}{k}$を$0$と$1$からなる$n$文字の文字列のうち、$1$が$k$個使われている文字列の集合とする。$\lambda=\lambda_1\cdots \lambda_n\in\binom{n}{k}$に対して置換$\sigma\in\mathfrak{S}_n$の$\binom{n}{k}$への作用を$\sigma\lambda=\lambda_{\inv{\sigma}(1)}\cdots\lambda_{\inv{\sigma}(n)}$で定める.
$\lambda=\lambda_1\cdots \lambda_n\in\binom{n}{k}$に対して、
\[
\Omega_\lambda=\set{V\in \text{Gr}_k(\complex^n)}{\dim(V\cap F_i)\geq \dim(\complex^\lambda\cap F_i),\quad \forall i\in \set{1,\cdots n}{}}
\]
をSchubert cellという. ここで、$\complex^\lambda=<e_{\lambda_1},\cdots,e_{\lambda_n}>$, $F_i=<e_{n-i+1},\cdots,e_n>$である. 

$\text{inv}(\lambda)=\set{(i,j)}{\lambda_i=1, \lambda_j=0, i<j}$とすると$\Omega_\lambda$は$\complex^{|\text{inv}(\lambda)|}$に同相であり
\[
\text{Gr}_k(\complex^n)=\bigsqcup_{\lambda\in\binom{n}{k}}\Omega_\lambda
\]
となる。したがって$H^*(\text{Gr}_k(\complex))$は$\Omega_\lambda$の定めるホモロジー類のPoincare双対$S_\lambda$たちで$\integer$上生成される. $\text{Gr}_k(\complex)$はequivariantly formalであるから$H^*_T(\text{Gr}_k(\complex^n))$は$H^*_T(\text{pt})=\integer[y_1,\cdots,y_n]$上$S_\lambda$たちで生成される.$S_\lambda$をSchubert classという。

したがって$2$つのSchubert classの積$S_\lambda S_\mu$はふたたび$\{S_\nu\}_{\nu\in\binom{n}{k}}$の$\integer[y_1,\cdots,y_n]$係数の線形結合
\[
S_\lambda S_\mu=\sum_{\nu\in\binom{n}{k}}C^\nu_{\lambda\mu}S_\nu
\]
で表すことができ、この時の係数$C^{\nu}_{\lambda\mu}$を計算する組み合わせ的手法を紹介することが本論文の目的である.


\subsection{GKM条件によるSchubert Classの特徴づけ}

$\text{Gr}_k(\complex^n)$の$T$作用における固定点は$\{\complex^\lambda\}_{\lambda\in\binom{n}{k}}$であるから、[GKM]より$H^*_T(\text{Gr}_k(\complex^n))$は$\bigoplus_{\lambda\in\binom{n}{k}}H^*_T(\text{pt})$の部分代数である。GKMの定理を適用するために$\text{Gr}_k(\complex^n)$の$T$不変な$\mathbb{CP}^1$を計算する.

\begin{prop}
  $\lambda,\mu\in\binom{n}{k}$に対して$\complex^\lambda$と$\complex^\mu$を結ぶ$T$不変な$\mathbb{CP}^1$が存在するための必要十分条件は, ある$(i, j)\in\text{inv}(\lambda)$に対して$\mu = (i, j)\lambda$
  となっていることである.
\end{prop}

\begin{proof}
  
\end{proof}




\subsection{Schubert puzzleによる方法}


\subsection{edge labeled tableuxによる方法}

$n$の分割$\lambda=(\lambda_1\geq\cdots\geq\lambda_k>0)$に対して、$1$行目に$\lambda_1$個の箱を, $2$行目に$\lambda_2$個の箱を,順に$k$行目まで左寄せで書いた図をYoung図形という。以降分割とYoung図形を同一視して同じ記号で表す. $\lambda$の各箱に、各行について左から右に狭義単調増大, 各列について上から下に広義単調増大となるように数字を書き入れたものをYoung tableauxという. 

\[
\ydiagram{4, 3, 3, 1},
\quad\begin{ytableau}
    1&1&2&2\\
    2&3&3\\
    3&4&4\\
    4
\end{ytableau}
\]

分割$\lambda, \mu$に対して、$\lambda>\mu\Leftrightarrow \lambda_i>\mu_i \forall i$によって順序を定める. $\mu$のYoung図形から$\lambda$に相当する部分を取り除いた図形を歪Young図形といい。$\mu/\lambda$で表す. 歪Young図形の各箱に、各行について左から右に狭義単調増大, 各列について上から下に広義単調増大となるように数字を書き入れたものをskew tableauxという. 
\[
\ydiagram{2+2, 2+1, 1+2, 1},
\quad\begin{ytableau}
    \none&\none&1&2\\
    \none&\none&2\\
    \none&1&3\\
    2
\end{ytableau}
\]





\subsection{weight preserving bijectionの構成}