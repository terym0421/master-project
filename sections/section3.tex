\section{同変Schubert計算}
\subsection{(同変/非同変)Schubert計算}
$\text{Gr}_k(\complex^n)=\set{V\subset \complex^n}{\dim V = k}$をGrassmann多様体という。$T=\complex^n$とするとき、$T$は$\complex^n$に
\[
(t_1,\cdots,t_n)\cdot(x_1,\cdots,x_n)=(t_1x_1,\cdots,t_nx_n)
\]
によって左から作用する。この作用は自然に$\text{Gr}_k(\complex^n)$への作用を誘導し、$\text{Gr}_k(\complex^n)$は$T$空間となる。$\text{Gr}_k(\complex^n)$の$T$同変コホモロジーの構造は組み合わせ的に決定することができる。




\subsection{GKM条件によるSchubert Classの特徴づけ}



\subsection{Schubert puzzleによる方法}

\subsection{edge labeled tableuxによる方法}

\subsection{weight preserving bijectionの構成}