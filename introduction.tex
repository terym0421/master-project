\section*{はじめに}

$2$つのSchubert classの積をSchubert classの線形結合で表した時の係数はLittlewood-Richardson数と呼ばれており,これを計算する組合せ論的手法がいくつも知られている(Young tableauxによる包括的な解説として\cite{fulton young tableaux}が挙げられる).同変コホモロジーにおいても,それぞれの$T$固定点に対応してequivariant Schubert classが存在し,同変コホモロジー環の基底となる.その場合のLittlewood-Richardson数は多項式となるが,本論文ではこれを計算する$2$つの手法を紹介する.$1$つはKnutson-Tao\cite{knutson tao}によって与えられたpuzzleによる方法で,\cite{positivity}の意味でのLittlewood-Richardson係数のpositivityを構成的に証明した.$2$つめはThomas-Yong\cite{thomas yong}によるedge labeled tableauxを用いた手法であり,Schützenbergerによるtableauxのje deu taquinを拡張した理論となっている.いずれも非同変な場合の自然な拡張になっており,特別な場合に非同変なLittlewood-Richardson数を計算することができる.

\cite{knutson tao}における証明は,GKMの定理に基づき,Littlewood-Richardson係数を特徴づける条件を導出することで証明している.\cite{thomas yong}においても,"Although one can biject the rule of Theorem 1.2
with earlier rules, $\cdots$"とあるように,\cite{knutson tao}のの手法などへの全単射を構成するのではなく,\cite{knutson tao}と同様の戦略で証明を行っている.本論文では\cite{thomas yong}で言及されたこの全単射を明示的に与えた(?).

本論文ではまず,セクション$1$において同変コホモロジーに関する基本的な事項を解説する.universal bundleを用いたBorel構成によるトポロジカルな定義と,de Rham複体の拡張であるWeil model/Cartan modelによる微分幾何学的な定義の両方を導入し,非同変のコホモロジーと同様,これら$2$つの定義がある意味で同値であること(同変de Rhamの定理)を紹介する.またAtiyah-Bottのlocalizationによって同変微分形式の積分が固定点における評価に帰着されるという重要な結果についても述べた.なお,セクション$1$の内容は$1.3$を除き\cite{tu equivariant}に従った.

次にセクション$2$において,同変Schubert計算における組合せ論の理論的基礎となるGoresky-Kottwitz-MacPhersonの定理(通称GKMの定理)を紹介する.この定理によって同変コホモロジー環は多項式環の直積の部分代数であることがわかり,特にトーラス不変な$1$次元軌道と固定点におけるisotropy表現の指標によって同変コホモロジー環が記述されることがわかる.後半では\cite{knutson tao}に従い,Grassmann多様体の場合に,トーラス同変コホモロジー環とその基底について詳しく紹介した.

最後にセクション$3$で同変Littlewood-Richardson係数を計算する手法として\cite{knutson tao}と\cite{thomas yong}による方法を紹介し,それら$2$つの同値性として,puzzleとedge labeled tableauxの間のweight preserving bijectionを構成する.